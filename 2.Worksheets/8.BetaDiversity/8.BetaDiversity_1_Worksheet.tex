% Options for packages loaded elsewhere
\PassOptionsToPackage{unicode}{hyperref}
\PassOptionsToPackage{hyphens}{url}
%
\documentclass[
]{article}
\usepackage{amsmath,amssymb}
\usepackage{lmodern}
\usepackage{ifxetex,ifluatex}
\ifnum 0\ifxetex 1\fi\ifluatex 1\fi=0 % if pdftex
  \usepackage[T1]{fontenc}
  \usepackage[utf8]{inputenc}
  \usepackage{textcomp} % provide euro and other symbols
\else % if luatex or xetex
  \usepackage{unicode-math}
  \defaultfontfeatures{Scale=MatchLowercase}
  \defaultfontfeatures[\rmfamily]{Ligatures=TeX,Scale=1}
\fi
% Use upquote if available, for straight quotes in verbatim environments
\IfFileExists{upquote.sty}{\usepackage{upquote}}{}
\IfFileExists{microtype.sty}{% use microtype if available
  \usepackage[]{microtype}
  \UseMicrotypeSet[protrusion]{basicmath} % disable protrusion for tt fonts
}{}
\makeatletter
\@ifundefined{KOMAClassName}{% if non-KOMA class
  \IfFileExists{parskip.sty}{%
    \usepackage{parskip}
  }{% else
    \setlength{\parindent}{0pt}
    \setlength{\parskip}{6pt plus 2pt minus 1pt}}
}{% if KOMA class
  \KOMAoptions{parskip=half}}
\makeatother
\usepackage{xcolor}
\IfFileExists{xurl.sty}{\usepackage{xurl}}{} % add URL line breaks if available
\IfFileExists{bookmark.sty}{\usepackage{bookmark}}{\usepackage{hyperref}}
\hypersetup{
  pdftitle={8. Worksheet: Among Site (Beta) Diversity -- Part 1},
  pdfauthor={Herbert Sizek; Z620: Quantitative Biodiversity, Indiana University},
  hidelinks,
  pdfcreator={LaTeX via pandoc}}
\urlstyle{same} % disable monospaced font for URLs
\usepackage[margin=2.54cm]{geometry}
\usepackage{color}
\usepackage{fancyvrb}
\newcommand{\VerbBar}{|}
\newcommand{\VERB}{\Verb[commandchars=\\\{\}]}
\DefineVerbatimEnvironment{Highlighting}{Verbatim}{commandchars=\\\{\}}
% Add ',fontsize=\small' for more characters per line
\usepackage{framed}
\definecolor{shadecolor}{RGB}{248,248,248}
\newenvironment{Shaded}{\begin{snugshade}}{\end{snugshade}}
\newcommand{\AlertTok}[1]{\textcolor[rgb]{0.94,0.16,0.16}{#1}}
\newcommand{\AnnotationTok}[1]{\textcolor[rgb]{0.56,0.35,0.01}{\textbf{\textit{#1}}}}
\newcommand{\AttributeTok}[1]{\textcolor[rgb]{0.77,0.63,0.00}{#1}}
\newcommand{\BaseNTok}[1]{\textcolor[rgb]{0.00,0.00,0.81}{#1}}
\newcommand{\BuiltInTok}[1]{#1}
\newcommand{\CharTok}[1]{\textcolor[rgb]{0.31,0.60,0.02}{#1}}
\newcommand{\CommentTok}[1]{\textcolor[rgb]{0.56,0.35,0.01}{\textit{#1}}}
\newcommand{\CommentVarTok}[1]{\textcolor[rgb]{0.56,0.35,0.01}{\textbf{\textit{#1}}}}
\newcommand{\ConstantTok}[1]{\textcolor[rgb]{0.00,0.00,0.00}{#1}}
\newcommand{\ControlFlowTok}[1]{\textcolor[rgb]{0.13,0.29,0.53}{\textbf{#1}}}
\newcommand{\DataTypeTok}[1]{\textcolor[rgb]{0.13,0.29,0.53}{#1}}
\newcommand{\DecValTok}[1]{\textcolor[rgb]{0.00,0.00,0.81}{#1}}
\newcommand{\DocumentationTok}[1]{\textcolor[rgb]{0.56,0.35,0.01}{\textbf{\textit{#1}}}}
\newcommand{\ErrorTok}[1]{\textcolor[rgb]{0.64,0.00,0.00}{\textbf{#1}}}
\newcommand{\ExtensionTok}[1]{#1}
\newcommand{\FloatTok}[1]{\textcolor[rgb]{0.00,0.00,0.81}{#1}}
\newcommand{\FunctionTok}[1]{\textcolor[rgb]{0.00,0.00,0.00}{#1}}
\newcommand{\ImportTok}[1]{#1}
\newcommand{\InformationTok}[1]{\textcolor[rgb]{0.56,0.35,0.01}{\textbf{\textit{#1}}}}
\newcommand{\KeywordTok}[1]{\textcolor[rgb]{0.13,0.29,0.53}{\textbf{#1}}}
\newcommand{\NormalTok}[1]{#1}
\newcommand{\OperatorTok}[1]{\textcolor[rgb]{0.81,0.36,0.00}{\textbf{#1}}}
\newcommand{\OtherTok}[1]{\textcolor[rgb]{0.56,0.35,0.01}{#1}}
\newcommand{\PreprocessorTok}[1]{\textcolor[rgb]{0.56,0.35,0.01}{\textit{#1}}}
\newcommand{\RegionMarkerTok}[1]{#1}
\newcommand{\SpecialCharTok}[1]{\textcolor[rgb]{0.00,0.00,0.00}{#1}}
\newcommand{\SpecialStringTok}[1]{\textcolor[rgb]{0.31,0.60,0.02}{#1}}
\newcommand{\StringTok}[1]{\textcolor[rgb]{0.31,0.60,0.02}{#1}}
\newcommand{\VariableTok}[1]{\textcolor[rgb]{0.00,0.00,0.00}{#1}}
\newcommand{\VerbatimStringTok}[1]{\textcolor[rgb]{0.31,0.60,0.02}{#1}}
\newcommand{\WarningTok}[1]{\textcolor[rgb]{0.56,0.35,0.01}{\textbf{\textit{#1}}}}
\usepackage{graphicx}
\makeatletter
\def\maxwidth{\ifdim\Gin@nat@width>\linewidth\linewidth\else\Gin@nat@width\fi}
\def\maxheight{\ifdim\Gin@nat@height>\textheight\textheight\else\Gin@nat@height\fi}
\makeatother
% Scale images if necessary, so that they will not overflow the page
% margins by default, and it is still possible to overwrite the defaults
% using explicit options in \includegraphics[width, height, ...]{}
\setkeys{Gin}{width=\maxwidth,height=\maxheight,keepaspectratio}
% Set default figure placement to htbp
\makeatletter
\def\fps@figure{htbp}
\makeatother
\setlength{\emergencystretch}{3em} % prevent overfull lines
\providecommand{\tightlist}{%
  \setlength{\itemsep}{0pt}\setlength{\parskip}{0pt}}
\setcounter{secnumdepth}{-\maxdimen} % remove section numbering
\ifluatex
  \usepackage{selnolig}  % disable illegal ligatures
\fi

\title{8. Worksheet: Among Site (Beta) Diversity -- Part 1}
\author{Herbert Sizek; Z620: Quantitative Biodiversity, Indiana
University}
\date{21 April, 2021}

\begin{document}
\maketitle

\hypertarget{overview}{%
\subsection{OVERVIEW}\label{overview}}

In this worksheet, we move beyond the investigation of within-site
\(\alpha\)-diversity. We will explore \(\beta\)-diversity, which is
defined as the diversity that occurs among sites. This requires that we
examine the compositional similarity of assemblages that vary in space
or time.

After completing this exercise you will know how to:

\begin{enumerate}
\def\labelenumi{\arabic{enumi}.}
\tightlist
\item
  formally quantify \(\beta\)-diversity
\item
  visualize \(\beta\)-diversity with heatmaps, cluster analysis, and
  ordination
\item
  test hypotheses about \(\beta\)-diversity using multivariate
  statistics
\end{enumerate}

\hypertarget{directions}{%
\subsection{Directions:}\label{directions}}

\begin{enumerate}
\def\labelenumi{\arabic{enumi}.}
\tightlist
\item
  In the Markdown version of this document in your cloned repo, change
  ``Student Name'' on line 3 (above) with your name.
\item
  Complete as much of the worksheet as possible during class.
\item
  Use the handout as a guide; it contains a more complete description of
  data sets along with examples of proper scripting needed to carry out
  the exercises.
\item
  Answer questions in the worksheet. Space for your answers is provided
  in this document and is indicated by the ``\textgreater{}'' character.
  If you need a second paragraph be sure to start the first line with
  ``\textgreater{}''. You should notice that the answer is highlighted
  in green by RStudio (color may vary if you changed the editor theme).
\item
  Before you leave the classroom today, it is \emph{imperative} that you
  \textbf{push} this file to your GitHub repo, at whatever stage you
  are. Ths will enable you to pull your work onto your own computer.
\item
  When you have completed the worksheet, \textbf{Knit} the text and code
  into a single PDF file by pressing the \texttt{Knit} button in the
  RStudio scripting panel. This will save the PDF output in your
  `8.BetaDiversity' folder.
\item
  After Knitting, please submit the worksheet by making a \textbf{push}
  to your GitHub repo and then create a \textbf{pull request} via
  GitHub. Your pull request should include this file
  (\textbf{8.BetaDiversity\_1\_Worksheet.Rmd}) with all code blocks
  filled out and questions answered) and the PDF output of
  \texttt{Knitr}\\
  (\textbf{8.BetaDiversity\_1\_Worksheet.pdf}).
\end{enumerate}

The completed exercise is due on \textbf{Friday, April
16\textsuperscript{th}, 2021 before 09:00 AM}.

\hypertarget{r-setup}{%
\subsection{1) R SETUP}\label{r-setup}}

Typically, the first thing you will do in either an R script or an
RMarkdown file is setup your environment. This includes things such as
setting the working directory and loading any packages that you will
need.

In the R code chunk below, provide the code to:

\begin{enumerate}
\def\labelenumi{\arabic{enumi}.}
\tightlist
\item
  clear your R environment,
\item
  print your current working directory,
\end{enumerate}

\begin{Shaded}
\begin{Highlighting}[]
\FunctionTok{rm}\NormalTok{(}\AttributeTok{list=}\FunctionTok{ls}\NormalTok{())}
\FunctionTok{getwd}\NormalTok{()}
\end{Highlighting}
\end{Shaded}

\begin{verbatim}
## [1] "D:/GitHub/QB2021_Sizek/2.Worksheets/8.BetaDiversity"
\end{verbatim}

\begin{Shaded}
\begin{Highlighting}[]
\FunctionTok{setwd}\NormalTok{(}\StringTok{"D:/GitHub/QB2021\_Sizek/2.Worksheets/8.BetaDiversity"}\NormalTok{)}
\end{Highlighting}
\end{Shaded}

\begin{enumerate}
\def\labelenumi{\arabic{enumi}.}
\setcounter{enumi}{2}
\tightlist
\item
  set your working directory to your ``\emph{/8.BetaDiversity}'' folder,
  and
\item
  load the \texttt{vegan} R package (be sure to install if needed).
\end{enumerate}

\begin{Shaded}
\begin{Highlighting}[]
\ControlFlowTok{for}\NormalTok{ (i  }\ControlFlowTok{in} \FunctionTok{c}\NormalTok{(}\StringTok{"vegan"}\NormalTok{,}\StringTok{"ade4"}\NormalTok{, }\StringTok{"viridis"}\NormalTok{, }\StringTok{"gplots"}\NormalTok{, }\StringTok{"BiodiversityR"}\NormalTok{, }\StringTok{"indicspecies"}\NormalTok{))\{}
  \FunctionTok{require}\NormalTok{(i,}\AttributeTok{character.only=}\ConstantTok{TRUE}\NormalTok{)}
\NormalTok{\}}
\end{Highlighting}
\end{Shaded}

\begin{verbatim}
## Loading required package: vegan
\end{verbatim}

\begin{verbatim}
## Warning: package 'vegan' was built under R version 3.6.3
\end{verbatim}

\begin{verbatim}
## Loading required package: permute
\end{verbatim}

\begin{verbatim}
## Warning: package 'permute' was built under R version 3.6.3
\end{verbatim}

\begin{verbatim}
## Loading required package: lattice
\end{verbatim}

\begin{verbatim}
## This is vegan 2.5-7
\end{verbatim}

\begin{verbatim}
## Loading required package: ade4
\end{verbatim}

\begin{verbatim}
## Warning: package 'ade4' was built under R version 3.6.3
\end{verbatim}

\begin{verbatim}
## Loading required package: viridis
\end{verbatim}

\begin{verbatim}
## Loading required package: viridisLite
\end{verbatim}

\begin{verbatim}
## Warning: package 'viridisLite' was built under R version 3.6.3
\end{verbatim}

\begin{verbatim}
## Loading required package: gplots
\end{verbatim}

\begin{verbatim}
## Warning: package 'gplots' was built under R version 3.6.3
\end{verbatim}

\begin{verbatim}
## 
## Attaching package: 'gplots'
\end{verbatim}

\begin{verbatim}
## The following object is masked from 'package:stats':
## 
##     lowess
\end{verbatim}

\begin{verbatim}
## Loading required package: BiodiversityR
\end{verbatim}

\begin{verbatim}
## Warning: package 'BiodiversityR' was built under R version 3.6.3
\end{verbatim}

\begin{verbatim}
## Loading required package: tcltk
\end{verbatim}

\begin{verbatim}
## Registered S3 methods overwritten by 'lme4':
##   method                          from
##   cooks.distance.influence.merMod car 
##   influence.merMod                car 
##   dfbeta.influence.merMod         car 
##   dfbetas.influence.merMod        car
\end{verbatim}

\begin{verbatim}
## BiodiversityR 2.12-3: Use command BiodiversityRGUI() to launch the Graphical User Interface; 
## to see changes use BiodiversityRGUI(changeLog=TRUE, backward.compatibility.messages=TRUE)
\end{verbatim}

\begin{verbatim}
## Loading required package: indicspecies
\end{verbatim}

\begin{verbatim}
## Warning: package 'indicspecies' was built under R version 3.6.3
\end{verbatim}

\hypertarget{loading-data}{%
\subsection{2) LOADING DATA}\label{loading-data}}

\hypertarget{load-dataset}{%
\subsubsection{Load dataset}\label{load-dataset}}

In the R code chunk below, do the following:

\begin{enumerate}
\def\labelenumi{\arabic{enumi}.}
\tightlist
\item
  load the \texttt{doubs} dataset from the \texttt{ade4} package, and
\item
  explore the structure of the dataset.
\end{enumerate}

\begin{Shaded}
\begin{Highlighting}[]
\CommentTok{\# note, pleae do not print the dataset when submitting}

\FunctionTok{data}\NormalTok{(doubs)}
\FunctionTok{summary}\NormalTok{(doubs)}
\end{Highlighting}
\end{Shaded}

\begin{verbatim}
##         Length Class      Mode
## env     11     data.frame list
## fish    27     data.frame list
## xy       2     data.frame list
## species  4     data.frame list
\end{verbatim}

\textbf{\emph{Question 1}}: Describe some of the attributes of the
\texttt{doubs} dataset.

\begin{enumerate}
\def\labelenumi{\alph{enumi}.}
\item
  How many objects are in \texttt{doubs}? \textgreater{}
  \textbf{\emph{Answer 1a}}: 4
\item
  How many fish species are there in the \texttt{doubs} dataset?
  \textgreater{} \textbf{\emph{Answer 1b}}: 27
\item
  How many sites are in the \texttt{doubs} dataset? \textgreater{}
  \textbf{\emph{Answer 1c}}: 30
\end{enumerate}

\hypertarget{visualizing-the-doubs-river-dataset}{%
\subsubsection{Visualizing the Doubs River
Dataset}\label{visualizing-the-doubs-river-dataset}}

\textbf{\emph{Question 2}}: Answer the following questions based on the
spatial patterns of richness (i.e., \(\alpha\)-diversity) and Brown
Trout (\emph{Salmo trutta}) abundance in the Doubs River.

\begin{enumerate}
\def\labelenumi{\alph{enumi}.}
\tightlist
\item
  How does fish richness vary along the sampled reach of the Doubs
  River? \textgreater{} \textbf{\emph{Answer 2a}}:
\end{enumerate}

\begin{Shaded}
\begin{Highlighting}[]
\NormalTok{S.obs }\OtherTok{\textless{}{-}} \ControlFlowTok{function}\NormalTok{(}\AttributeTok{x=}\StringTok{\textquotesingle{}\textquotesingle{}}\NormalTok{)\{}
  \FunctionTok{rowSums}\NormalTok{(x}\SpecialCharTok{\textgreater{}}\DecValTok{0}\NormalTok{)}\SpecialCharTok{*}\DecValTok{1}
\NormalTok{\}}
\NormalTok{cFisher }\OtherTok{\textless{}{-}}\ControlFlowTok{function}\NormalTok{(}\AttributeTok{x=}\StringTok{\textquotesingle{}\textquotesingle{}}\NormalTok{)\{}
  \FunctionTok{fisher.alpha}\NormalTok{(}\FunctionTok{as.integer}\NormalTok{(}\FunctionTok{as.vector}\NormalTok{(x[x}\SpecialCharTok{\textgreater{}}\DecValTok{0}\NormalTok{])))}
\NormalTok{\}}
\FunctionTok{S.obs}\NormalTok{(doubs}\SpecialCharTok{$}\NormalTok{fish)}
\end{Highlighting}
\end{Shaded}

\begin{verbatim}
##  1  2  3  4  5  6  7  8  9 10 11 12 13 14 15 16 17 18 19 20 21 22 23 24 25 26 
##  1  3  4  8 11 10  5  0  5  6  6  6  6 10 11 17 22 23 23 22 23 22  3  8  8 21 
## 27 28 29 30 
## 22 22 26 21
\end{verbatim}

\begin{Shaded}
\begin{Highlighting}[]
\FunctionTok{apply}\NormalTok{(doubs}\SpecialCharTok{$}\NormalTok{fish,}\DecValTok{1}\NormalTok{,cFisher)}
\end{Highlighting}
\end{Shaded}

\begin{verbatim}
##          1          2          3          4          5          6          7 
##  0.5252543  1.2838726  1.7118462  4.7170847  5.6420718  7.4791167  2.4967465 
##          8          9         10         11         12         13         14 
##  1.0000000  2.7823859  3.9775884  5.4028053  3.1515726  3.0201644  5.5647670 
##         15         16         17         18         19         20         21 
##  5.7778860 11.1701802 17.5099120 20.8366287 18.3058149 13.3546197 13.2359384 
##         22         23         24         25         26         27         28 
## 10.8014143  5.4525555  6.9657122 13.1934641 16.2109829 12.0091137 11.0334863 
##         29         30 
## 12.5581380  8.6723803
\end{verbatim}

\begin{enumerate}
\def\labelenumi{\alph{enumi}.}
\setcounter{enumi}{1}
\item
  How does Brown Trout (\emph{Salmo trutta}) abundance vary along the
  sampled reach of the Doubs River? \textgreater{} \textbf{\emph{Answer
  2b}}: There are regions where there is higher abundance and regions
  with lower abundance, low - medium - low - medium - high - low - high
\item
  What do these patterns say about the limitations of using richness
  when examining patterns of biodiversity?
\end{enumerate}

\begin{quote}
\textbf{\emph{Answer 2c}}: There is spatial variation in richness across
space, so if your habitat of interest is heterogeneous the samples might
reflect the habitat health or some other variable rather than a large
variation in species richness.
\end{quote}

\hypertarget{quantifying-beta-diversity}{%
\subsection{3) QUANTIFYING
BETA-DIVERSITY}\label{quantifying-beta-diversity}}

In the R code chunk below, do the following:

\begin{enumerate}
\def\labelenumi{\arabic{enumi}.}
\tightlist
\item
  write a function (\texttt{beta.w()}) to calculate Whittaker's
  \(\beta\)-diversity (i.e., \(\beta_{w}\)) that accepts a
  site-by-species matrix with optional arguments to specify pairwise
  turnover between two sites, and
\end{enumerate}

\begin{Shaded}
\begin{Highlighting}[]
\NormalTok{beta.w }\OtherTok{\textless{}{-}} \ControlFlowTok{function}\NormalTok{(}\AttributeTok{site.by.species =} \StringTok{""}\NormalTok{,}\AttributeTok{sitenum1=}\StringTok{""}\NormalTok{,}\AttributeTok{sitenum2=}\StringTok{""}\NormalTok{)\{}
  \ControlFlowTok{if}\NormalTok{ (sitenum1}\SpecialCharTok{==}\StringTok{""} \SpecialCharTok{|}\NormalTok{ sitenum2}\SpecialCharTok{==}\StringTok{""}\NormalTok{)\{}
    \ControlFlowTok{if}\NormalTok{ (}\FunctionTok{xor}\NormalTok{(sitenum1}\SpecialCharTok{==}\StringTok{""}\NormalTok{,sitenum2}\SpecialCharTok{==}\StringTok{""}\NormalTok{))\{}
      \FunctionTok{warning}\NormalTok{(}\StringTok{"beta.w: One site was specified, returning beta for entire SbyS matrix"}\NormalTok{)}
\NormalTok{    \}}
\NormalTok{    SbyS.pa }\OtherTok{\textless{}{-}} \FunctionTok{decostand}\NormalTok{(site.by.species,}\AttributeTok{method =} \StringTok{"pa"}\NormalTok{) }\CommentTok{\# convert to presence absence}
\NormalTok{    S }\OtherTok{\textless{}{-}} \FunctionTok{ncol}\NormalTok{(SbyS.pa[,}\FunctionTok{which}\NormalTok{(}\FunctionTok{colSums}\NormalTok{(SbyS.pa)}\SpecialCharTok{\textgreater{}}\DecValTok{0}\NormalTok{)])}
\NormalTok{    a.bar}\OtherTok{\textless{}{-}}\FunctionTok{mean}\NormalTok{(}\FunctionTok{specnumber}\NormalTok{(SbyS.pa))}
\NormalTok{    b.w }\OtherTok{\textless{}{-}}\FunctionTok{round}\NormalTok{(S}\SpecialCharTok{/}\NormalTok{a.bar,}\DecValTok{3}\NormalTok{)}
    \FunctionTok{return}\NormalTok{(b.w)}
\NormalTok{  \}}
  \ControlFlowTok{else}\NormalTok{\{}
\NormalTok{    site1 }\OtherTok{=}\NormalTok{ site.by.species[sitenum1,]}
\NormalTok{    site2 }\OtherTok{=}\NormalTok{ site.by.species[sitenum2,]}
\NormalTok{    site1 }\OtherTok{=} \FunctionTok{subset}\NormalTok{(site1, }\AttributeTok{select =}\NormalTok{ site1}\SpecialCharTok{\textgreater{}}\DecValTok{0}\NormalTok{)}
\NormalTok{    site2 }\OtherTok{=} \FunctionTok{subset}\NormalTok{(site2, }\AttributeTok{select =}\NormalTok{ site2}\SpecialCharTok{\textgreater{}}\DecValTok{0}\NormalTok{)}
\NormalTok{    gamma }\OtherTok{=} \FunctionTok{union}\NormalTok{(}\FunctionTok{colnames}\NormalTok{(site1),}\FunctionTok{colnames}\NormalTok{(site2))}
\NormalTok{    s }\OtherTok{=} \FunctionTok{length}\NormalTok{(gamma)}
\NormalTok{    a.bar }\OtherTok{=} \FunctionTok{mean}\NormalTok{(}\FunctionTok{c}\NormalTok{(}\FunctionTok{specnumber}\NormalTok{(site1),}\FunctionTok{specnumber}\NormalTok{(site2)))}
\NormalTok{    b.w }\OtherTok{=} \FunctionTok{round}\NormalTok{(s}\SpecialCharTok{/}\NormalTok{a.bar }\SpecialCharTok{{-}}\DecValTok{1}\NormalTok{,}\DecValTok{3}\NormalTok{)}
    \FunctionTok{return}\NormalTok{(b.w)}
\NormalTok{  \}}
  
\NormalTok{\}}
\end{Highlighting}
\end{Shaded}

\begin{enumerate}
\def\labelenumi{\arabic{enumi}.}
\setcounter{enumi}{1}
\tightlist
\item
  use this function to analyze various aspects of \(\beta\)-diversity in
  the Doubs River.
\end{enumerate}

\begin{Shaded}
\begin{Highlighting}[]
\FunctionTok{beta.w}\NormalTok{(doubs}\SpecialCharTok{$}\NormalTok{fish)}
\end{Highlighting}
\end{Shaded}

\begin{verbatim}
## [1] 2.16
\end{verbatim}

\begin{Shaded}
\begin{Highlighting}[]
\FunctionTok{beta.w}\NormalTok{(doubs}\SpecialCharTok{$}\NormalTok{fish,}\DecValTok{5}\NormalTok{,}\DecValTok{26}\NormalTok{)}
\end{Highlighting}
\end{Shaded}

\begin{verbatim}
## [1] 0.438
\end{verbatim}

\textbf{\emph{Question 3}}: Using your \texttt{beta.w()} function above,
answer the following questions:

\begin{enumerate}
\def\labelenumi{\alph{enumi}.}
\item
  Describe how local richness (\(\alpha\)) and turnover (\(\beta\))
  contribute to regional (\(\gamma\)) fish diversity in the Doubs.
  \textgreater{} *\textbf{Answer 3a}: This question is a bit odd,
  because these three are in a mathamatical relation, being: \$\gamma =
  \alpha \beta \$, for at least whitaker's \(\beta\) diversity. The two
  components that are directly measured are \(\alpha\) and \(\gamma\),
  with the prior being the mean local richness (average number of
  species per site) and the latter being the total number of species
  across all sites. This means that as \(beta\) approaches 1, the
  species are distributed randomly across sites, that is most sites have
  the same compositions of species. If there is wide variation between
  species at sites, \(\gamma\) will increase while \(\alpha\) could
  remain the same, increasing \(\beta\). If the distribution of species
  in sites increases then \(\alpha\) will fall, increasing \(\beta\).
\item
  Is the fish assemblage at site 1 more similar to the one at site 2 or
  site 10? \textgreater{} \textbf{\emph{Answer 3b}}:
\end{enumerate}

\begin{Shaded}
\begin{Highlighting}[]
\FunctionTok{S.obs}\NormalTok{(doubs}\SpecialCharTok{$}\NormalTok{fish[}\FunctionTok{c}\NormalTok{(}\DecValTok{1}\NormalTok{,}\DecValTok{2}\NormalTok{,}\DecValTok{10}\NormalTok{),])}
\end{Highlighting}
\end{Shaded}

\begin{verbatim}
##  1  2 10 
##  1  3  6
\end{verbatim}

\begin{Shaded}
\begin{Highlighting}[]
\FunctionTok{beta.w}\NormalTok{(doubs}\SpecialCharTok{$}\NormalTok{fish,}\DecValTok{1}\NormalTok{,}\DecValTok{2}\NormalTok{)}
\end{Highlighting}
\end{Shaded}

\begin{verbatim}
## [1] 0.5
\end{verbatim}

\begin{Shaded}
\begin{Highlighting}[]
\FunctionTok{beta.w}\NormalTok{(doubs}\SpecialCharTok{$}\NormalTok{fish,}\DecValTok{1}\NormalTok{,}\DecValTok{10}\NormalTok{)}
\end{Highlighting}
\end{Shaded}

\begin{verbatim}
## [1] 0.714
\end{verbatim}

\begin{quote}
The species at site 1 is more similar to the species at site 2, but this
is primarily driven by that site 2 has fewer species than site 10.
\end{quote}

\begin{enumerate}
\def\labelenumi{\alph{enumi}.}
\setcounter{enumi}{2}
\tightlist
\item
  Using your understanding of the equation
  \(\beta_{w} = \gamma/\alpha\), how would your interpretation of
  \(\beta\) change if we instead defined beta additively (i.e.,
  \(\beta = \gamma - \alpha\))? \textgreater{} \textbf{\emph{Answer
  3c}}: This would make the measure a bit harder for me to understand
  value in as the number of sites would effect the outcome more. It
  would be more like a measure of the first moment of species. I don't
  know really how I would apply it thoughtfully.
\end{enumerate}

\hypertarget{the-resemblance-matrix}{%
\subsubsection{The Resemblance Matrix}\label{the-resemblance-matrix}}

In order to quantify \(\beta\)-diversity for more than two samples, we
need to introduce a new primary ecological data structure: the
\textbf{Resemblance Matrix}.

\textbf{\emph{Question 4}}: How do incidence- and abundance-based
metrics differ in their treatment of rare species?

\begin{quote}
\textbf{\emph{Answer 4}}: Incidence matrices are binary matricies while
abundance matrices are whole numbers. This means that rare species are
of equal value in binary matrices while they are not in abundance
matrices.
\end{quote}

In the R code chunk below, do the following:

\begin{enumerate}
\def\labelenumi{\arabic{enumi}.}
\tightlist
\item
  make a new object, \texttt{fish}, containing the fish abundance data
  for the Doubs River,
\item
  remove any sites where no fish were observed (i.e., rows with sum of
  zero),
\end{enumerate}

\begin{Shaded}
\begin{Highlighting}[]
\NormalTok{fish }\OtherTok{\textless{}{-}}\NormalTok{ doubs}\SpecialCharTok{$}\NormalTok{fish}
\NormalTok{fish}\OtherTok{\textless{}{-}}\NormalTok{ fish[}\SpecialCharTok{{-}}\FunctionTok{which}\NormalTok{(}\FunctionTok{rowSums}\NormalTok{(fish)}\SpecialCharTok{==}\DecValTok{0}\NormalTok{),] }\CommentTok{\#remove rows with zero values. }
\end{Highlighting}
\end{Shaded}

\begin{enumerate}
\def\labelenumi{\arabic{enumi}.}
\setcounter{enumi}{2}
\tightlist
\item
  construct a resemblance matrix based on Sørensen's Similarity
  (``fish.ds''), and
\end{enumerate}

\begin{Shaded}
\begin{Highlighting}[]
\NormalTok{fish.ds }\OtherTok{\textless{}{-}} \FunctionTok{vegdist}\NormalTok{(fish,}\AttributeTok{method=}\StringTok{"bray"}\NormalTok{,}\AttributeTok{binary=}\ConstantTok{TRUE}\NormalTok{)}
\end{Highlighting}
\end{Shaded}

\begin{enumerate}
\def\labelenumi{\arabic{enumi}.}
\setcounter{enumi}{3}
\tightlist
\item
  construct a resemblance matrix based on Bray-Curtis Distance
  (``fish.db'').
\end{enumerate}

\begin{Shaded}
\begin{Highlighting}[]
\NormalTok{fish.db }\OtherTok{\textless{}{-}} \FunctionTok{vegdist}\NormalTok{(fish,}\AttributeTok{method=}\StringTok{"bray"}\NormalTok{)}
\end{Highlighting}
\end{Shaded}

\textbf{\emph{Question 5}}: Using the distance matrices from above,
answer the following questions:

\begin{enumerate}
\def\labelenumi{\alph{enumi}.}
\tightlist
\item
  Does the resemblance matrix (\texttt{fish.db}) represent similarity or
  dissimilarity? What information in the resemblance matrix led you to
  arrive at your answer? \textgreater{} \textbf{\emph{Answer 5a}}:
  Dissimilarity is closer to 1, because the values close to the diagnal
  are near zero, while farther away (farther away sites) are near 1.
\end{enumerate}

\begin{Shaded}
\begin{Highlighting}[]
\NormalTok{order }\OtherTok{\textless{}{-}} \FunctionTok{rev}\NormalTok{(}\FunctionTok{attr}\NormalTok{(fish.db,}\StringTok{"Labels"}\NormalTok{))}
\FunctionTok{levelplot}\NormalTok{(}\FunctionTok{as.matrix}\NormalTok{(fish.db)[,order],}\AttributeTok{aspect=}\StringTok{"iso"}\NormalTok{,}\AttributeTok{col.regions=}\NormalTok{inferno,}\AttributeTok{xlab=}\StringTok{"Doubs Site"}\NormalTok{,}\AttributeTok{ylab=}\StringTok{"Doubs Site"}\NormalTok{, }\AttributeTok{main=}\StringTok{"Bray{-}Curtis Distance"}\NormalTok{,}\AttributeTok{scales=}\FunctionTok{list}\NormalTok{(}\AttributeTok{cex=}\FloatTok{0.5}\NormalTok{))}
\end{Highlighting}
\end{Shaded}

\includegraphics{8.BetaDiversity_1_Worksheet_files/figure-latex/unnamed-chunk-11-1.pdf}

\begin{enumerate}
\def\labelenumi{\alph{enumi}.}
\setcounter{enumi}{1}
\tightlist
\item
  Compare the resemblance matrices (\texttt{fish.db} or
  \texttt{fish.ds}) you just created. How does the choice of the
  Sørensen or Bray-Curtis distance influence your interpretation of site
  (dis)similarity?
\end{enumerate}

\begin{quote}
\textbf{\emph{Answer 5b}}: They are fairly similar, Sorensen makes it
seem that the sites are more homogenous then Bray-Curtis. Sorensen
generally provides lower values (sites are more similar) in comparison
to Bray-Curtis.
\end{quote}

\begin{Shaded}
\begin{Highlighting}[]
\NormalTok{order }\OtherTok{\textless{}{-}} \FunctionTok{rev}\NormalTok{(}\FunctionTok{attr}\NormalTok{(fish.ds,}\StringTok{"Labels"}\NormalTok{))}
\FunctionTok{levelplot}\NormalTok{(}\FunctionTok{as.matrix}\NormalTok{(fish.ds)[,order],}\AttributeTok{aspect=}\StringTok{"iso"}\NormalTok{,}\AttributeTok{col.regions=}\NormalTok{inferno,}\AttributeTok{xlab=}\StringTok{"Doubs Site"}\NormalTok{,}\AttributeTok{ylab=}\StringTok{"Doubs Site"}\NormalTok{, }\AttributeTok{main=}\StringTok{"Sorensen Distance"}\NormalTok{,}\AttributeTok{scales=}\FunctionTok{list}\NormalTok{(}\AttributeTok{cex=}\FloatTok{0.5}\NormalTok{))}
\end{Highlighting}
\end{Shaded}

\includegraphics{8.BetaDiversity_1_Worksheet_files/figure-latex/unnamed-chunk-12-1.pdf}

\begin{Shaded}
\begin{Highlighting}[]
\NormalTok{order }\OtherTok{\textless{}{-}} \FunctionTok{rev}\NormalTok{(}\FunctionTok{attr}\NormalTok{(fish.ds}\SpecialCharTok{{-}}\NormalTok{fish.db,}\StringTok{"Labels"}\NormalTok{))}
\FunctionTok{levelplot}\NormalTok{(}\FunctionTok{as.matrix}\NormalTok{(fish.ds}\SpecialCharTok{{-}}\NormalTok{fish.db)[,order],}\AttributeTok{aspect=}\StringTok{"iso"}\NormalTok{,}\AttributeTok{col.regions=}\NormalTok{inferno,}\AttributeTok{xlab=}\StringTok{"Doubs Site"}\NormalTok{,}\AttributeTok{ylab=}\StringTok{"Doubs Site"}\NormalTok{, }\AttributeTok{main=}\StringTok{"Sorensen minus Bray{-}Curtis Distance"}\NormalTok{,}\AttributeTok{scales=}\FunctionTok{list}\NormalTok{(}\AttributeTok{cex=}\FloatTok{0.5}\NormalTok{))}
\end{Highlighting}
\end{Shaded}

\includegraphics{8.BetaDiversity_1_Worksheet_files/figure-latex/unnamed-chunk-13-1.pdf}

\hypertarget{visualizing-beta-diversity}{%
\subsection{4) VISUALIZING
BETA-DIVERSITY}\label{visualizing-beta-diversity}}

\hypertarget{a.-heatmaps}{%
\subsubsection{A. Heatmaps}\label{a.-heatmaps}}

In the R code chunk below, do the following:

\begin{enumerate}
\def\labelenumi{\arabic{enumi}.}
\item
  define a color palette, \textgreater{} Viridis was imported above
\item
  define the order of sites in the Doubs River, and
\item
  use the \texttt{levelplot()} function to create a heatmap of fish
  abundances in the Doubs River.
\end{enumerate}

\begin{Shaded}
\begin{Highlighting}[]
\NormalTok{order }\OtherTok{\textless{}{-}} \FunctionTok{rev}\NormalTok{(}\FunctionTok{attr}\NormalTok{(fish.db,}\StringTok{"Labels"}\NormalTok{))}
\FunctionTok{levelplot}\NormalTok{(}\FunctionTok{as.matrix}\NormalTok{(fish.db)[,order],}\AttributeTok{aspect=}\StringTok{"iso"}\NormalTok{,}\AttributeTok{col.regions=}\NormalTok{viridis,}\AttributeTok{xlab=}\StringTok{"Doubs Site"}\NormalTok{,}\AttributeTok{ylab=}\StringTok{"Doubs Site"}\NormalTok{, }\AttributeTok{main=}\StringTok{"Bray{-}Curtis Distance"}\NormalTok{,}\AttributeTok{scales=}\FunctionTok{list}\NormalTok{(}\AttributeTok{cex=}\FloatTok{0.5}\NormalTok{))}
\end{Highlighting}
\end{Shaded}

\includegraphics{8.BetaDiversity_1_Worksheet_files/figure-latex/unnamed-chunk-14-1.pdf}

\hypertarget{b.-cluster-analysis}{%
\subsubsection{B. Cluster Analysis}\label{b.-cluster-analysis}}

In the R code chunk below, do the following:

\begin{enumerate}
\def\labelenumi{\arabic{enumi}.}
\tightlist
\item
  perform a cluster analysis using Ward's Clustering, and
\item
  plot your cluster analysis (use either \texttt{hclust} or
  \texttt{heatmap.2}).
\end{enumerate}

\begin{Shaded}
\begin{Highlighting}[]
\NormalTok{fish.ward }\OtherTok{\textless{}{-}} \FunctionTok{hclust}\NormalTok{(fish.db, }\AttributeTok{method =} \StringTok{"ward.D2"}\NormalTok{)}
\FunctionTok{par}\NormalTok{(}\AttributeTok{mar =} \FunctionTok{c}\NormalTok{(}\DecValTok{1}\NormalTok{,}\DecValTok{5}\NormalTok{,}\DecValTok{2}\NormalTok{,}\DecValTok{2}\NormalTok{)}\SpecialCharTok{+}\FloatTok{0.1}\NormalTok{)}
\FunctionTok{plot}\NormalTok{(fish.ward,}\AttributeTok{main =} \StringTok{"Doubs River Fish: Ward Clustering"}\NormalTok{,}\AttributeTok{ylab=}\StringTok{"Squared Bray{-}Curtis Distance"}\NormalTok{)}
\end{Highlighting}
\end{Shaded}

\includegraphics{8.BetaDiversity_1_Worksheet_files/figure-latex/unnamed-chunk-15-1.pdf}

\textbf{\emph{Question 6}}: Based on cluster analyses and the
introductory plots that we generated after loading the data, develop an
ecological hypothesis for fish diversity the \texttt{doubs} data set?

\begin{quote}
\textbf{\emph{Answer 6}}: There are two general habitats, one that is
upstream (1-14) and downstream (15 -30), the lower habitat is could be
broken down into three pieces, but I would want to test against other
clustering algorithms.
\end{quote}

\hypertarget{c.-ordination}{%
\subsubsection{C. Ordination}\label{c.-ordination}}

\hypertarget{principal-coordinates-analysis-pcoa}{%
\subsubsection{Principal Coordinates Analysis
(PCoA)}\label{principal-coordinates-analysis-pcoa}}

In the R code chunk below, do the following:

\begin{enumerate}
\def\labelenumi{\arabic{enumi}.}
\tightlist
\item
  perform a Principal Coordinates Analysis to visualize beta-diversity
\item
  calculate the variation explained by the first three axes in your
  ordination
\end{enumerate}

\begin{Shaded}
\begin{Highlighting}[]
\NormalTok{fish.pcoa }\OtherTok{\textless{}{-}} \FunctionTok{cmdscale}\NormalTok{(fish.db, }\AttributeTok{eig =}\ConstantTok{TRUE}\NormalTok{, }\AttributeTok{k=}\DecValTok{3}\NormalTok{)}
\NormalTok{var1 }\OtherTok{\textless{}{-}} \FunctionTok{round}\NormalTok{(fish.pcoa}\SpecialCharTok{$}\NormalTok{eig[}\DecValTok{1}\NormalTok{]}\SpecialCharTok{/}\FunctionTok{sum}\NormalTok{(fish.pcoa}\SpecialCharTok{$}\NormalTok{eig),}\DecValTok{3}\NormalTok{)}
\NormalTok{var2 }\OtherTok{\textless{}{-}} \FunctionTok{round}\NormalTok{(fish.pcoa}\SpecialCharTok{$}\NormalTok{eig[}\DecValTok{2}\NormalTok{]}\SpecialCharTok{/}\FunctionTok{sum}\NormalTok{(fish.pcoa}\SpecialCharTok{$}\NormalTok{eig),}\DecValTok{3}\NormalTok{)}
\NormalTok{var3 }\OtherTok{\textless{}{-}} \FunctionTok{round}\NormalTok{(fish.pcoa}\SpecialCharTok{$}\NormalTok{eig[}\DecValTok{3}\NormalTok{]}\SpecialCharTok{/}\FunctionTok{sum}\NormalTok{(fish.pcoa}\SpecialCharTok{$}\NormalTok{eig),}\DecValTok{3}\NormalTok{)}
\NormalTok{sum.eig }\OtherTok{\textless{}{-}} \FunctionTok{sum}\NormalTok{(var1,var2,var3)}
\end{Highlighting}
\end{Shaded}

\begin{enumerate}
\def\labelenumi{\arabic{enumi}.}
\setcounter{enumi}{2}
\tightlist
\item
  plot the PCoA ordination,
\item
  label the sites as points using the Doubs River site number, and
\item
  identify influential species and add species coordinates to PCoA plot.
\end{enumerate}

\begin{Shaded}
\begin{Highlighting}[]
\NormalTok{fishREL }\OtherTok{\textless{}{-}}\NormalTok{ fish}
\ControlFlowTok{for}\NormalTok{ (i }\ControlFlowTok{in} \DecValTok{1}\SpecialCharTok{:}\FunctionTok{nrow}\NormalTok{(fish))\{}
\NormalTok{  fishREL[i,] }\OtherTok{=}\NormalTok{ fish[i,]}\SpecialCharTok{/}\FunctionTok{sum}\NormalTok{(fish[i,])}
\NormalTok{\}}

\NormalTok{fish.pcoa }\OtherTok{\textless{}{-}}\FunctionTok{add.spec.scores}\NormalTok{(fish.pcoa,fishREL,}\AttributeTok{method=}\StringTok{"pcoa.scores"}\NormalTok{)}
\end{Highlighting}
\end{Shaded}

\begin{Shaded}
\begin{Highlighting}[]
\FunctionTok{par}\NormalTok{(}\AttributeTok{mar=}\FunctionTok{c}\NormalTok{(}\DecValTok{5}\NormalTok{,}\DecValTok{5}\NormalTok{,}\DecValTok{1}\NormalTok{,}\DecValTok{2}\NormalTok{)}\SpecialCharTok{+}\FloatTok{0.1}\NormalTok{)}
\FunctionTok{plot}\NormalTok{(fish.pcoa}\SpecialCharTok{$}\NormalTok{points[,}\DecValTok{1}\NormalTok{],fish.pcoa}\SpecialCharTok{$}\NormalTok{points[,}\DecValTok{2}\NormalTok{],}\AttributeTok{ylim=}\FunctionTok{c}\NormalTok{(}\SpecialCharTok{{-}}\FloatTok{0.2}\NormalTok{,}\FloatTok{0.7}\NormalTok{),}
     \AttributeTok{xlab =} \FunctionTok{paste}\NormalTok{(}\StringTok{"PCoA 1("}\NormalTok{,var1}\SpecialCharTok{*}\DecValTok{100}\NormalTok{,}\StringTok{"\%)"}\NormalTok{,}\AttributeTok{sep=}\StringTok{\textquotesingle{}\textquotesingle{}}\NormalTok{),}
     \AttributeTok{ylab =} \FunctionTok{paste}\NormalTok{(}\StringTok{"PCoA 2("}\NormalTok{,var2}\SpecialCharTok{*}\DecValTok{100}\NormalTok{,}\StringTok{"\%)"}\NormalTok{,}\AttributeTok{sep=}\StringTok{\textquotesingle{}\textquotesingle{}}\NormalTok{),}
     \AttributeTok{pch =}\DecValTok{16}\NormalTok{ ,}\AttributeTok{cex =}\FloatTok{2.0}\NormalTok{,}\AttributeTok{type =}\StringTok{\textquotesingle{}n\textquotesingle{}}\NormalTok{,}\AttributeTok{cex.lab=}\FloatTok{1.5}\NormalTok{,}\AttributeTok{cex.axis=}\FloatTok{1.2}\NormalTok{,}\AttributeTok{axes=}\ConstantTok{FALSE}
\NormalTok{     )}

\FunctionTok{axis}\NormalTok{(}\AttributeTok{side =}\DecValTok{1}\NormalTok{, }\AttributeTok{labels=}\ConstantTok{TRUE}\NormalTok{,}\AttributeTok{lwd.ticks=}\DecValTok{2}\NormalTok{,}\AttributeTok{cex.axis =}\FloatTok{1.2}\NormalTok{,}\AttributeTok{las=}\DecValTok{1}\NormalTok{)}
\FunctionTok{axis}\NormalTok{(}\AttributeTok{side =}\DecValTok{2}\NormalTok{, }\AttributeTok{labels=}\ConstantTok{TRUE}\NormalTok{,}\AttributeTok{lwd.ticks=}\DecValTok{2}\NormalTok{,}\AttributeTok{cex.axis =}\FloatTok{1.2}\NormalTok{,}\AttributeTok{las=}\DecValTok{1}\NormalTok{)}
\FunctionTok{abline}\NormalTok{(}\AttributeTok{h=}\DecValTok{0}\NormalTok{,}\AttributeTok{v=}\DecValTok{0}\NormalTok{,}\AttributeTok{lty=}\DecValTok{3}\NormalTok{)}
\FunctionTok{box}\NormalTok{(}\AttributeTok{lwd=}\DecValTok{2}\NormalTok{)}
\FunctionTok{points}\NormalTok{(fish.pcoa}\SpecialCharTok{$}\NormalTok{points[,}\DecValTok{1}\NormalTok{],fish.pcoa}\SpecialCharTok{$}\NormalTok{points[,}\DecValTok{2}\NormalTok{],}
       \AttributeTok{pch =}\DecValTok{19}\NormalTok{, }\AttributeTok{cex =}\DecValTok{3}\NormalTok{, }\AttributeTok{bg=}\StringTok{\textquotesingle{}gray\textquotesingle{}}\NormalTok{,}\AttributeTok{col=}\StringTok{\textquotesingle{}gray\textquotesingle{}}\NormalTok{)}
\FunctionTok{text}\NormalTok{(fish.pcoa}\SpecialCharTok{$}\NormalTok{points[,}\DecValTok{1}\NormalTok{],fish.pcoa}\SpecialCharTok{$}\NormalTok{points[,}\DecValTok{2}\NormalTok{],}
     \AttributeTok{labels =}\FunctionTok{row.names}\NormalTok{(fish.pcoa}\SpecialCharTok{$}\NormalTok{points))}

\FunctionTok{text}\NormalTok{(fish.pcoa}\SpecialCharTok{$}\NormalTok{cproj[,}\DecValTok{1}\NormalTok{], fish.pcoa}\SpecialCharTok{$}\NormalTok{cproj[,}\DecValTok{2}\NormalTok{], }\AttributeTok{labels =} \FunctionTok{row.names}\NormalTok{(fish.pcoa}\SpecialCharTok{$}\NormalTok{cproj),}\AttributeTok{col=}\StringTok{"black"}\NormalTok{)}
\end{Highlighting}
\end{Shaded}

\includegraphics{8.BetaDiversity_1_Worksheet_files/figure-latex/unnamed-chunk-18-1.pdf}

In the R code chunk below, do the following:

\begin{enumerate}
\def\labelenumi{\arabic{enumi}.}
\tightlist
\item
  identify influential species based on correlations along each PCoA
  axis (use a cutoff of 0.70), and
\item
  use a permutation test (999 permutations) to test the correlations of
  each species along each axis.
\end{enumerate}

\begin{Shaded}
\begin{Highlighting}[]
\NormalTok{spe.corr }\OtherTok{\textless{}{-}} \FunctionTok{add.spec.scores}\NormalTok{(fish.pcoa,fishREL,}\AttributeTok{method =} \StringTok{"cor.scores"}\NormalTok{)}\SpecialCharTok{$}\NormalTok{cproj}
\NormalTok{corr.cut }\OtherTok{\textless{}{-}} \FloatTok{0.7}
\NormalTok{imp.spp }\OtherTok{\textless{}{-}}\NormalTok{spe.corr[}\FunctionTok{abs}\NormalTok{(spe.corr[,}\DecValTok{1}\NormalTok{])}\SpecialCharTok{\textgreater{}=}\NormalTok{corr.cut }\SpecialCharTok{|} \FunctionTok{abs}\NormalTok{(spe.corr[,}\DecValTok{2}\NormalTok{])}\SpecialCharTok{\textgreater{}=}\NormalTok{ corr.cut,]}
\NormalTok{fit }\OtherTok{\textless{}{-}}\FunctionTok{envfit}\NormalTok{(fish.pcoa,fishREL,}\AttributeTok{perm=}\DecValTok{999}\NormalTok{)}
\end{Highlighting}
\end{Shaded}

\textbf{\emph{Question 7}}: Address the following questions about the
ordination results of the \texttt{doubs} data set:

\begin{enumerate}
\def\labelenumi{\alph{enumi}.}
\item
  Describe the grouping of sites in the Doubs River based on fish
  community composition. \textgreater{} \textbf{\emph{Answer 7a}}:
  Because the PCoA is separating out based on the same principles as
  Ward Clustering (correlation levels between groups), just with
  slightly different methods, we see the same results, where there is
  the general two groups of 1-14 and 15-30 with the exception that 15 is
  similar to 1-14 on the first two PCs. The separation along the first
  PC appears mainly to be driven by the presence or absence of three
  species.
\item
  Generate a hypothesis about which fish species are potential
  indicators of river quality. \textgreater{} \textbf{\emph{Answer 7b}}:
  This question is too subjective. How do you define river quality? For
  example, if we are selecting for local diveristy, we might say that
  Alal is deterimental to the river quality, but it could also be that
  Alal is more constrained to a specific habitat or are expert niche
  constructors and exclude other species, but might influence terrestial
  species differently. We would first have to define a metric for river
  quality.
\end{enumerate}

\hypertarget{synthesis}{%
\subsection{SYNTHESIS}\label{synthesis}}

Using the \texttt{mobsim} package from the DataWrangling module last
week, simulate two local communities each containing 1000 individuals
(\emph{N}) and 25 species (\emph{S}), but with one having a random
spatial distribution and the other having a patchy spatial distribution.

\begin{Shaded}
\begin{Highlighting}[]
\FunctionTok{require}\NormalTok{(mobsim)}
\end{Highlighting}
\end{Shaded}

\begin{verbatim}
## Loading required package: mobsim
\end{verbatim}

\begin{verbatim}
## Warning: package 'mobsim' was built under R version 3.6.3
\end{verbatim}

\begin{Shaded}
\begin{Highlighting}[]
\NormalTok{comA }\OtherTok{\textless{}{-}} \FunctionTok{sim\_poisson\_community}\NormalTok{(}\AttributeTok{s\_pool =} \DecValTok{25}\NormalTok{, }\AttributeTok{n\_sim =} \DecValTok{1000}\NormalTok{, }\AttributeTok{sad\_type =} \StringTok{"lnorm"}\NormalTok{, }
        \AttributeTok{sad\_coef =} \FunctionTok{list}\NormalTok{(}\StringTok{"meanlog"} \OtherTok{=} \DecValTok{2}\NormalTok{, }\StringTok{"sdlog"} \OtherTok{=} \DecValTok{1}\NormalTok{))}

\NormalTok{comB }\OtherTok{\textless{}{-}} \FunctionTok{sim\_thomas\_community}\NormalTok{(}\AttributeTok{s\_pool =} \DecValTok{25}\NormalTok{, }\AttributeTok{n\_sim =} \DecValTok{1000}\NormalTok{, }\AttributeTok{sad\_type =} \StringTok{"lnorm"}\NormalTok{, }
        \AttributeTok{sad\_coef =} \FunctionTok{list}\NormalTok{(}\StringTok{"meanlog"} \OtherTok{=} \DecValTok{2}\NormalTok{, }\StringTok{"sdlog"} \OtherTok{=} \DecValTok{1}\NormalTok{))}
\end{Highlighting}
\end{Shaded}

Take ten (10) subsamples from each site using the quadrat function and
answer the following questions:

\begin{Shaded}
\begin{Highlighting}[]
\NormalTok{comm\_matA }\OtherTok{\textless{}{-}} \FunctionTok{sample\_quadrats}\NormalTok{(comA, }\AttributeTok{n\_quadrats =} \DecValTok{10}\NormalTok{, }\AttributeTok{quadrat\_area =} \FloatTok{0.03}\NormalTok{, }
             \AttributeTok{avoid\_overlap =}\NormalTok{ T, }\AttributeTok{plot=}\NormalTok{F)}
\end{Highlighting}
\end{Shaded}

\begin{verbatim}
## Warning in sample_quadrats(comA, n_quadrats = 10, quadrat_area = 0.03, avoid_overlap = T, : Cannot find a sampling layout with no overlap.
##                                             Install the package spatstat for an improved method for non-overlapping squares,
##                                             Use less quadrats or smaller quadrat area, or set avoid_overlap to FALSE.
\end{verbatim}

\begin{Shaded}
\begin{Highlighting}[]
\NormalTok{comm\_matB }\OtherTok{\textless{}{-}} \FunctionTok{sample\_quadrats}\NormalTok{(comB, }\AttributeTok{n\_quadrats =} \DecValTok{10}\NormalTok{, }\AttributeTok{quadrat\_area =} \FloatTok{0.03}\NormalTok{, }
             \AttributeTok{avoid\_overlap =}\NormalTok{ T, }\AttributeTok{plot=}\NormalTok{F)}
\end{Highlighting}
\end{Shaded}

\begin{verbatim}
## Warning in sample_quadrats(comB, n_quadrats = 10, quadrat_area = 0.03, avoid_overlap = T, : Cannot find a sampling layout with no overlap.
##                                             Install the package spatstat for an improved method for non-overlapping squares,
##                                             Use less quadrats or smaller quadrat area, or set avoid_overlap to FALSE.
\end{verbatim}

\begin{enumerate}
\def\labelenumi{\arabic{enumi})}
\tightlist
\item
  Compare the average pairwise similarity among subsamples in site 1
  (random spatial distribution) to the average pairswise similarity
  among subsamples in site 2 (patchy spatial distribution).
\end{enumerate}

\begin{Shaded}
\begin{Highlighting}[]
\NormalTok{comA.ds }\OtherTok{\textless{}{-}} \FunctionTok{vegdist}\NormalTok{(comm\_matA}\SpecialCharTok{$}\NormalTok{spec\_dat,}\AttributeTok{method=}\StringTok{"bray"}\NormalTok{,}\AttributeTok{binary=}\ConstantTok{TRUE}\NormalTok{)}
\NormalTok{comB.ds }\OtherTok{\textless{}{-}} \FunctionTok{vegdist}\NormalTok{(comm\_matB}\SpecialCharTok{$}\NormalTok{spec\_dat,}\AttributeTok{method=}\StringTok{"bray"}\NormalTok{,}\AttributeTok{binary=}\ConstantTok{TRUE}\NormalTok{)}
\NormalTok{comA.db }\OtherTok{\textless{}{-}} \FunctionTok{vegdist}\NormalTok{(comm\_matA}\SpecialCharTok{$}\NormalTok{spec\_dat,}\AttributeTok{method=}\StringTok{"bray"}\NormalTok{)}
\NormalTok{comB.db }\OtherTok{\textless{}{-}} \FunctionTok{vegdist}\NormalTok{(comm\_matB}\SpecialCharTok{$}\NormalTok{spec\_dat,}\AttributeTok{method=}\StringTok{"bray"}\NormalTok{)}
\FunctionTok{print}\NormalTok{(}\FunctionTok{paste}\NormalTok{(}\StringTok{"Random Community\textquotesingle{}s Average Similarity (Sorensen index):"}\NormalTok{,}\FunctionTok{mean}\NormalTok{(comA.ds)))}
\end{Highlighting}
\end{Shaded}

\begin{verbatim}
## [1] "Random Community's Average Similarity (Sorensen index): 0.327179651063905"
\end{verbatim}

\begin{Shaded}
\begin{Highlighting}[]
\FunctionTok{print}\NormalTok{(}\FunctionTok{paste}\NormalTok{(}\StringTok{"Patchy Community\textquotesingle{}s Average Similarity (Sorensen index):"}\NormalTok{,}\FunctionTok{mean}\NormalTok{(comB.ds)))}
\end{Highlighting}
\end{Shaded}

\begin{verbatim}
## [1] "Patchy Community's Average Similarity (Sorensen index): 0.740014480798795"
\end{verbatim}

Use a t-test to determine whether compositional similarity was affected
by the spatial distribution. \textgreater First we need to look at if
the t-test is appropriate in this case.

\begin{Shaded}
\begin{Highlighting}[]
\FunctionTok{hist}\NormalTok{(comA.ds,}\AttributeTok{breaks=}\DecValTok{10}\NormalTok{)}
\end{Highlighting}
\end{Shaded}

\includegraphics{8.BetaDiversity_1_Worksheet_files/figure-latex/unnamed-chunk-23-1.pdf}

\begin{Shaded}
\begin{Highlighting}[]
\FunctionTok{hist}\NormalTok{(comB.ds,}\AttributeTok{breaks=}\DecValTok{10}\NormalTok{)}
\end{Highlighting}
\end{Shaded}

\includegraphics{8.BetaDiversity_1_Worksheet_files/figure-latex/unnamed-chunk-23-2.pdf}

\begin{quote}
Doesn't look like it from the historgrams because the limit of Sorensen
index caps the values. To confirm, an F-test will do.
\end{quote}

\begin{Shaded}
\begin{Highlighting}[]
\FunctionTok{var.test}\NormalTok{(comA.ds,comB.ds)}
\end{Highlighting}
\end{Shaded}

\begin{verbatim}
## 
##  F test to compare two variances
## 
## data:  comA.ds and comB.ds
## F = 0.21009, num df = 44, denom df = 44, p-value = 8.817e-07
## alternative hypothesis: true ratio of variances is not equal to 1
## 95 percent confidence interval:
##  0.1154527 0.3823023
## sample estimates:
## ratio of variances 
##          0.2100901
\end{verbatim}

\begin{quote}
The F-test says that the variances between the datasets are different
meaning that the t-test is not appropriate. Let's repeat for BC index.
\end{quote}

\begin{Shaded}
\begin{Highlighting}[]
\FunctionTok{hist}\NormalTok{(comA.db,}\AttributeTok{breaks=}\DecValTok{10}\NormalTok{)}
\end{Highlighting}
\end{Shaded}

\includegraphics{8.BetaDiversity_1_Worksheet_files/figure-latex/unnamed-chunk-25-1.pdf}

\begin{Shaded}
\begin{Highlighting}[]
\FunctionTok{hist}\NormalTok{(comB.db,}\AttributeTok{breaks=}\DecValTok{10}\NormalTok{)}
\end{Highlighting}
\end{Shaded}

\includegraphics{8.BetaDiversity_1_Worksheet_files/figure-latex/unnamed-chunk-25-2.pdf}

\begin{Shaded}
\begin{Highlighting}[]
\FunctionTok{var.test}\NormalTok{(comA.db,comB.db)}
\end{Highlighting}
\end{Shaded}

\begin{verbatim}
## 
##  F test to compare two variances
## 
## data:  comA.db and comB.db
## F = 0.32775, num df = 44, denom df = 44, p-value = 0.0003306
## alternative hypothesis: true ratio of variances is not equal to 1
## 95 percent confidence interval:
##  0.1801110 0.5964076
## sample estimates:
## ratio of variances 
##          0.3277493
\end{verbatim}

\begin{quote}
Again we see that the t-test is inappropriate. Probably some other test
could be used but I don't know frequentist statistics that well. From
looking at the histograms, the pairwise comparisons look very different,
with the clustered communities having more communities that are
completely different from one another.
\end{quote}

Finally, compare the compositional similarity of site 1 and site 2 to
the source community?

\begin{Shaded}
\begin{Highlighting}[]
\NormalTok{RACcomA }\OtherTok{\textless{}{-}} \FunctionTok{rad.lognormal}\NormalTok{(comA}\SpecialCharTok{$}\NormalTok{census)}
\end{Highlighting}
\end{Shaded}

\begin{verbatim}
## Warning in Ops.factor(left, right): '>' not meaningful for factors
\end{verbatim}

\begin{verbatim}
## Error in y + 0.1 : non-numeric argument to binary operator
\end{verbatim}

\begin{Shaded}
\begin{Highlighting}[]
\FunctionTok{plot}\NormalTok{(RACcomA, }\AttributeTok{main =} \StringTok{"Random community"}\NormalTok{)}
\end{Highlighting}
\end{Shaded}

\includegraphics{8.BetaDiversity_1_Worksheet_files/figure-latex/unnamed-chunk-26-1.pdf}

\begin{Shaded}
\begin{Highlighting}[]
\NormalTok{RACcomAsample }\OtherTok{\textless{}{-}} \FunctionTok{rad.lognormal}\NormalTok{(comm\_matA}\SpecialCharTok{$}\NormalTok{spec\_dat)}
\FunctionTok{plot}\NormalTok{(RACcomAsample, }\AttributeTok{main =} \StringTok{"Random community sample"}\NormalTok{)}
\end{Highlighting}
\end{Shaded}

\includegraphics{8.BetaDiversity_1_Worksheet_files/figure-latex/unnamed-chunk-26-2.pdf}

\begin{Shaded}
\begin{Highlighting}[]
\NormalTok{RACcomB }\OtherTok{\textless{}{-}} \FunctionTok{rad.lognormal}\NormalTok{(comB}\SpecialCharTok{$}\NormalTok{census)}
\end{Highlighting}
\end{Shaded}

\begin{verbatim}
## Warning in Ops.factor(left, right): '>' not meaningful for factors
\end{verbatim}

\begin{verbatim}
## Error in y + 0.1 : non-numeric argument to binary operator
\end{verbatim}

\begin{Shaded}
\begin{Highlighting}[]
\FunctionTok{plot}\NormalTok{(RACcomB, }\AttributeTok{main =} \StringTok{"Clustered community"}\NormalTok{)}
\end{Highlighting}
\end{Shaded}

\includegraphics{8.BetaDiversity_1_Worksheet_files/figure-latex/unnamed-chunk-27-1.pdf}

\begin{Shaded}
\begin{Highlighting}[]
\NormalTok{RACcomBsample }\OtherTok{\textless{}{-}} \FunctionTok{rad.lognormal}\NormalTok{(comm\_matB}\SpecialCharTok{$}\NormalTok{spec\_dat)}
\FunctionTok{plot}\NormalTok{(RACcomBsample, }\AttributeTok{main =} \StringTok{"Clustered community sample"}\NormalTok{)}
\end{Highlighting}
\end{Shaded}

\includegraphics{8.BetaDiversity_1_Worksheet_files/figure-latex/unnamed-chunk-27-2.pdf}

\begin{quote}
We see that the rank abundance for the actual communities are most
likely a geometric/power law distribution. With the sampling we see that
the probability of observing a single observation of a species is much
lower in the clustering case and that there are fewer species observed
in total, this makes sense because of how the communities are generated.
\end{quote}

\begin{enumerate}
\def\labelenumi{\arabic{enumi})}
\setcounter{enumi}{1}
\tightlist
\item
  Create a cluster diagram or ordination using your simulated data. Are
  there any visual trends that would suggest a difference in composition
  between site 1 and site 2? Describe.
\end{enumerate}

\begin{Shaded}
\begin{Highlighting}[]
\NormalTok{comA.ward }\OtherTok{\textless{}{-}} \FunctionTok{hclust}\NormalTok{(comA.db, }\AttributeTok{method =} \StringTok{"ward.D2"}\NormalTok{)}
\FunctionTok{par}\NormalTok{(}\AttributeTok{mar =} \FunctionTok{c}\NormalTok{(}\DecValTok{1}\NormalTok{,}\DecValTok{5}\NormalTok{,}\DecValTok{2}\NormalTok{,}\DecValTok{2}\NormalTok{)}\SpecialCharTok{+}\FloatTok{0.1}\NormalTok{)}
\FunctionTok{plot}\NormalTok{(comA.ward,}\AttributeTok{main =} \StringTok{"Random Communities: Ward Clustering"}\NormalTok{,}\AttributeTok{ylab=}\StringTok{"Squared Bray{-}Curtis Distance"}\NormalTok{)}
\end{Highlighting}
\end{Shaded}

\includegraphics{8.BetaDiversity_1_Worksheet_files/figure-latex/unnamed-chunk-28-1.pdf}

\begin{Shaded}
\begin{Highlighting}[]
\NormalTok{comB.ward }\OtherTok{\textless{}{-}} \FunctionTok{hclust}\NormalTok{(comB.db, }\AttributeTok{method =} \StringTok{"ward.D2"}\NormalTok{)}
\FunctionTok{par}\NormalTok{(}\AttributeTok{mar =} \FunctionTok{c}\NormalTok{(}\DecValTok{1}\NormalTok{,}\DecValTok{5}\NormalTok{,}\DecValTok{2}\NormalTok{,}\DecValTok{2}\NormalTok{)}\SpecialCharTok{+}\FloatTok{0.1}\NormalTok{)}
\FunctionTok{plot}\NormalTok{(comB.ward,}\AttributeTok{main =} \StringTok{"Clusered Communities: Ward Clustering"}\NormalTok{,}\AttributeTok{ylab=}\StringTok{"Squared Bray{-}Curtis Distance"}\NormalTok{)}
\end{Highlighting}
\end{Shaded}

\includegraphics{8.BetaDiversity_1_Worksheet_files/figure-latex/unnamed-chunk-29-1.pdf}

\begin{quote}
We see that by looking at the axis values, that the clustered
communities are farther apart from one another in composition than the
random communities, though it does appear that we have some communities
that are close to one another. It would probably be good to do other
followup on this, such as does the BC distance correlate to other
aspects, such as the Euclidian distance between sites.
\end{quote}

\end{document}
